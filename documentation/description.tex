 \section{Описание}

Функция \texttt{supersvd} по двум заданным наборам пространственно--временных полей строит матрицу ковариации, а затем вычисляет её неполное сингулярное разложение. 

Функция \texttt{supersvd} принимает на вход 2 обязательных аргумента (два поля, максимально скоррелированные моды которых мы ищем) и 2 опциональных параметра: количество пар максимально скоррелированных мод (по умолчанию 3) и значение переключателя режима вычитания из поля его среднего по времени значения (по умолчанию \texttt{True}, то есть из поля \textit{вычитается} его среднее по времени значение).

Пусть $X(t), Y(t)$  --- два меняющихся во времени поля, максимально скоррелированные моды которых мы ищем, причём $\operatorname{dim}(X)=nT \times nX$\footnote{Здесь и далее размерности массивов указаны в порядке, принятом в \texttt{C} и \texttt{Python}. В \texttt{Fortran} размерности массивов следует развернуть в обратном порядке} и $\operatorname{dim}(Y)\hm=nT \times nY$, где $nX$ и $nY$ могут быть одним или несколькими измерениями массивов (в случае среднемесячных данных INMCM $nX$ и $nY$ --- $120 \times 180$). 
Функция \texttt{supersvd} вычисляет разложение вида:
\begin{equation}
\begin{aligned}
X(t) &= \overline{X} + XV_1 XC_1(t) + XV_2 XC_2(t) + \ldots + XV_k XC_k(t) + \ldots,\\
Y(t) &= \overline{Y} + YV_1 YC_1(t) + YV_2 YC_2(t) + \ldots + YV_k YC_k(t) + \ldots,
\end{aligned}
\label{eq:svd}
\end{equation}
где 
\begin{equation*}
\overline{X}=\frac{1}{nT}\sum_{t=1}^{nT} X(t),\qquad
\overline{Y}=\frac{1}{nT}\sum_{t=1}^{nT} Y(t),
\end{equation*}
а $k$ --- количество пар максимально скоррелированных мод. В \eqref{eq:svd} каждое новое слагаемое получается максимизацией корреляции между $XC_k(t)$ и $YC_k(t)$, а
$XV_k, YV_k$~--- два семейства ортогональных пространственных мод.

Моды $XV_k, YV_k$ являются левыми и правыми сингулярными векторами матрицы ковариации $$C =\frac{1}{nT} \sum_{t=1}^{nT} (X(t) - \overline{X}) (Y(t) - \overline{Y})^{\mathsf T}.$$

Функция \texttt{supersvd} возвращает: 
\begin{itemize}
	\item массивы \texttt{x\_coeff}, \texttt{y\_coeff}  временных коэффициентов $XC(t), YC(t)$ разложения \eqref{eq:svd} \big(раз\-мерности $k \times nT$\big);
	\item массив \texttt{x\_vect}  левых сингулярных векторов $XV$ \big(размерности $k \times nX$\big);
	\item массив \texttt{y\_vect} правых сингулярных векторов $YV$ \big(размерности $k \times nY$\big);
	\item массив \texttt{corrcoeff}, содержащий $k$ коэффициентов корреляции между $XC_k(t)$ и  $YC_k(t)$;
	\item массив \texttt{x\_variance\_fraction} (\texttt{y\_variance\_fraction}), содержащий доли дисперсии, приходящиеся на каждый из $k$ левых (правых) сингулярных векторов;
	\item массив \texttt{eigenvalue\_fraction}, содержащий долю дисперсии матрицы ковариации, приходящуюся на $k$-ую пару сингулярных векторов;
	\item массив \texttt{eigenvalues} сингулярных значений матрицы ковариации $C$.   
\end{itemize}
