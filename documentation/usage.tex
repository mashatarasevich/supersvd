\section{Использование}

Функция \texttt{supersvd} может вызываться как и из другой \texttt{Python}-функции, принимая на вход массивы данных, так и из командной строки, принимая на вход бинарные файлы (\texttt{.STD}). Последняя возможность реализована в функции \texttt{main}.

 Функция \texttt{main} принимает на вход 3 обязательных аргумента: 
 \begin{itemize}
 	\item[\texttt{-x}] имя файла, содержащего первое из полей (например, \texttt{X.STD});
 	\item[\texttt{-y}] имя файла, содержащего второе из полей  (например, \texttt{Y.STD})\footnote{Если нужно посчитать ЭОФы, то в качестве первого и второго нужно задать одно и то же поле, то есть передать два раза имя одного файла.};
 	\item[\texttt{-t}, \texttt{-{}-time}] длину временного интервала (например, в случае среднемесячных данных исторического эксперимента с INMCM это 165 лет).
 \end{itemize}
 
 Также функция \texttt{main} принимает 7 необязательных (опциональных) параметров:
 \begin{itemize}
 	\item[\texttt{-{}-type}] тип используемых данных --- \texttt{real} (4 байта) или \texttt{double} (8 байт), значение по умолчанию~--- \texttt{real};
 	\item[\texttt{-k}] количество вычисляемых пар максимально скоррелированных мод, значение по умолчанию --- 3;
 	\item[\texttt{-xv}] имя файла, в который запишется массив \texttt{x\_vect};
 	\item[\texttt{-yv}] имя файла, в который запишется массив \texttt{y\_vect};
 	\item[\texttt{-xc}] имя файла, в который запишется массив \texttt{x\_coeff};
 	\item[\texttt{-yc}] имя файла, в который запишется массив \texttt{y\_coeff};
 	\item[\texttt{-stat}] имя файла (предпочтительно в формате \texttt{.CSV}), в который для каждого $k$ запишутся домноженные на 100\% элементы массивов: \texttt{corrcoeff}, \texttt{x\_variance\_fraction}, \texttt{y\_vari\-ance\_fraction}, \texttt{eigenvalue\_fraction}. 
 \end{itemize}  

Функция \texttt{main} также может быть запущена с ключом \texttt{-{}-dont-subtract-mean}: при этом из полей $X$, $Y$ \textit{не будут вычитаться} их средние по времени значения.

Итак, чтобы вычислить с помощью функции \texttt{main} 4 максимально скоррелированные моды аномалий температуры и давления (типа \texttt{float}) и сохранить все возможные результаты, достаточно в командной строке выполнить:\\
\verb|python3 main.py -x ts.std -y ps.std -t 1147 -k 4 -xv tsv.std -yv psv.std|\\
\verb|                 -xc tsc.std -yc psc.std -stat c.csv|

Информацию, сохраняемую в файл \texttt{c.csv}, функция \texttt{main} также выводит на экран.