\documentclass[12pt, a4paper]{article}
\usepackage[T2A]{fontenc}
\usepackage[utf8]{inputenc}
\usepackage[russian]{babel}
\usepackage[top=15mm,left=17.5mm, right=17.5mm, bottom=20mm]{geometry}
\usepackage{amsmath}
\usepackage{icomma}

\newcommand*{\hm}[1]{#1\nobreak\discretionary{}%
	{\hbox{$\mathsurround=0pt #1$}}{}}

\title{\large Программа для вычисления максимально скоррелированных мод и ЭОФ}
\author{Мария Тарасевич, \texttt{mashatarasevich@gmail.com}}
\date{}

\begin{document}
\maketitle
	
	
Написанная на языке \texttt{Python 3} программа состоит из двух файлов: \texttt{main.py} и \texttt{supersvd.py}. В файле \texttt{supersvd.py} находится алгоритм вычисления максимально скоррелированных мод, а в \texttt{main.py} --- вспомогательный код, который анализирует ключи запуска программы, делает чтение входных данных из файлов, а также записывает в выходные файлы результаты работы алгоритма.

Функцию \texttt{supersvd} можно напрямую использовать из кода на \texttt{Python}, в этом случае не обязательно сохранять массивы в виде файлов на диске. 
 
 \section{Описание}

Функция \texttt{supersvd} по двум заданным наборам пространственно--временных полей строит матрицу ковариации, а затем вычисляет её неполное сингулярное разложение. 

Функция \texttt{supersvd} принимает на вход 2 обязательных аргумента (два поля, максимально скоррелированные моды которых мы ищем) и 2 опциональных параметра: количество пар максимально скоррелированных мод (по умолчанию 3) и значение переключателя режима вычитания из поля его среднего по времени значения (по умолчанию \texttt{True}, то есть из поля \textit{вычитается} его среднее по времени значение).

Пусть $X(t), Y(t)$  --- два меняющихся во времени поля, максимально скоррелированные моды которых мы ищем, причём $\operatorname{dim}(X)=nT \times nX$\footnote{Здесь и далее размерности массивов указаны в порядке, принятом в \texttt{C} и \texttt{Python}. В \texttt{Fortran} размерности массивов следует развернуть в обратном порядке} и $\operatorname{dim}(Y)\hm=nT \times nY$, где $nX$ и $nY$ могут быть одним или несколькими измерениями массивов (в случае среднемесячных данных INMCM $nX$ и $nY$ --- $120 \times 180$). 
Функция \texttt{supersvd} вычисляет разложение вида:
\begin{equation}
\begin{aligned}
X(t) &= \overline{X} + XV_1 XC_1(t) + XV_2 XC_2(t) + \ldots + XV_k XC_k(t) + \ldots,\\
Y(t) &= \overline{Y} + YV_1 YC_1(t) + YV_2 YC_2(t) + \ldots + YV_k YC_k(t) + \ldots,
\end{aligned}
\label{eq:svd}
\end{equation}
где 
\begin{equation*}
\overline{X}=\frac{1}{nT}\sum_{t=1}^{nT} X(t),\qquad
\overline{Y}=\frac{1}{nT}\sum_{t=1}^{nT} Y(t),
\end{equation*}
а $k$ --- количество пар максимально скоррелированных мод. В \eqref{eq:svd} каждое новое слагаемое получается максимизацией корреляции между $XC_k(t)$ и $YC_k(t)$, а
$XV_k, YV_k$~--- два семейства ортогональных пространственных мод.

Моды $XV_k, YV_k$ являются левыми и правыми сингулярными векторами матрицы ковариации $$C =\frac{1}{nT} \sum_{t=1}^{nT} (X(t) - \overline{X}) (Y(t) - \overline{Y})^{\mathsf T}.$$

Функция \texttt{supersvd} возвращает: 
\begin{itemize}
	\item массивы \texttt{x\_coeff}, \texttt{y\_coeff}  временных коэффициентов $XC(t), YC(t)$ разложения \eqref{eq:svd} \big(раз\-мерности $k \times nT$\big);
	\item массив \texttt{x\_vect}  левых сингулярных векторов $XV$ \big(размерности $k \times nX$\big);
	\item массив \texttt{y\_vect} правых сингулярных векторов $YV$ \big(размерности $k \times nY$\big);
	\item массив \texttt{corrcoeff}, содержащий $k$ коэффициентов корреляции между $XC_k(t)$ и  $YC_k(t)$;
	\item массив \texttt{x\_variance\_fraction} (\texttt{y\_variance\_fraction}), содержащий доли дисперсии, приходящиеся на каждый из $k$ левых (правых) сингулярных векторов;
	\item массив \texttt{eigenvalue\_fraction}, содержащий долю дисперсии матрицы ковариации, приходящуюся на $k$-ую пару сингулярных векторов;
	\item массив \texttt{eigenvalues} сингулярных значений матрицы ковариации $C$.   
\end{itemize}


\section{Использование}

Функция \texttt{supersvd} может вызываться как и из другой \texttt{Python}-функции, принимая на вход массивы данных, так и из командной строки, принимая на вход бинарные файлы (\texttt{.STD}). Последняя возможность реализована в функции \texttt{main}.

 Функция \texttt{main} принимает на вход 3 обязательных аргумента: 
 \begin{itemize}
 	\item[\texttt{-x}] имя файла, содержащего первое из полей (например, \texttt{X.STD});
 	\item[\texttt{-y}] имя файла, содержащего второе из полей  (например, \texttt{Y.STD})\footnote{Если нужно посчитать ЭОФы, то в качестве первого и второго нужно задать одно и то же поле, то есть передать два раза имя одного файла.};
 	\item[\texttt{-t}, \texttt{-{}-time}] длину временного интервала (например, в случае среднемесячных данных исторического эксперимента с INMCM это 165 лет).
 \end{itemize}
 
 Также функция \texttt{main} принимает 7 необязательных (опциональных) параметров:
 \begin{itemize}
 	\item[\texttt{-{}-type}] тип используемых данных --- \texttt{real} (4 байта) или \texttt{double} (8 байт), значение по умолчанию~--- \texttt{real};
 	\item[\texttt{-k}] количество вычисляемых пар максимально скоррелированных мод, значение по умолчанию --- 3;
 	\item[\texttt{-xv}] имя файла, в который запишется массив \texttt{x\_vect};
 	\item[\texttt{-yv}] имя файла, в который запишется массив \texttt{y\_vect};
 	\item[\texttt{-xc}] имя файла, в который запишется массив \texttt{x\_coeff};
 	\item[\texttt{-yc}] имя файла, в который запишется массив \texttt{y\_coeff};
 	\item[\texttt{-stat}] имя файла (предпочтительно в формате \texttt{.CSV}), в который для каждого $k$ запишутся домноженные на 100\% элементы массивов: \texttt{corrcoeff}, \texttt{x\_variance\_fraction}, \texttt{y\_vari\-ance\_fraction}, \texttt{eigenvalue\_fraction}. 
 \end{itemize}  

Функция \texttt{main} также может быть запущена с ключом \texttt{-{}-dont-subtract-mean}: при этом из полей $X$, $Y$ \textit{не будут вычитаться} их средние по времени значения.

Итак, чтобы вычислить с помощью функции \texttt{main} 4 максимально скоррелированные моды аномалий температуры и давления (типа \texttt{float}) и сохранить все возможные результаты, достаточно в командной строке выполнить:\\
\verb|python3 main.py -x ts.std -y ps.std -t 1147 -k 4 -xv tsv.std -yv psv.std|\\
\verb|                 -xc tsc.std -yc psc.std -stat c.csv|

Информацию, сохраняемую в файл \texttt{c.csv}, функция \texttt{main} также выводит на экран.

\end{document}
